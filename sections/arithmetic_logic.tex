\section{Arithmetic and Logical Operations}
\subsection{Basic operations}
\begin{table}[h]
    \centering
    \small
    \renewcommand{\arraystretch}{1.2}
    \caption{Common arithmetic and logical instructions in x86-64. The \texttt{leaq} 
    (Load Effective Address) instruction is commonly used to perform simple arithmetic. 
    The remaining ones are more standard unary or binary operations. We use the notation 
    \texttt{>>} for right shifts, with signed values using \textit{arithmetic right shift} 
    (`SAR`) and unsigned values using \textit{logical right shift} (`SHR`). Logical right 
    shift puts ZEROS to the beginning while Arithmetic right shift puts the SIGN BIT (0 or 1)
     to the beginning. Additionally, all of them set the condition flags except \texttt{lea}.}
    \label{basic-arith}
    \begin{tabular}{l l l}
        \toprule
        \textbf{Instruction} & \textbf{Effect} & \textbf{Description} \\
        \midrule
        \texttt{leaq} $S, D$  & $D \leftarrow \&S$ & Load effective address \\
        \midrule
        \texttt{INC} $D$  & $D \leftarrow D+1$ & Increment \\
        \texttt{DEC} $D$  & $D \leftarrow D-1$ & Decrement \\
        \texttt{NEG} $D$  & $D \leftarrow -D$ & Negate \\
        \texttt{NOT} $D$  & $D \leftarrow \sim D$ & Complement \\
        \midrule
        \texttt{ADD} $S, D$  & $D \leftarrow D + S$ & Add \\
        \texttt{SUB} $S, D$  & $D \leftarrow D - S$ & Subtract \\
        \texttt{IMUL} $S, D$  & $D \leftarrow D * S$ & Multiply \\
        \texttt{XOR} $S, D$  & $D \leftarrow D  \texttt{\string^} S$ & Exclusive-or \\
        \texttt{OR} $S, D$  & $D \leftarrow D | S$ & Or \\
        \texttt{AND} $S, D$  & $D \leftarrow D \& S$ & And \\
        \midrule
        \texttt{SAL} $k, D$  & $D \leftarrow D << k$ & Left shift \\
        \texttt{SHL} $k, D$  & $D \leftarrow D << k$ & Left shift (same as \texttt{SAL}) \\
        \texttt{SAR} $k, D$  & $D \leftarrow D >>_A k$ & Arithmetic right shift \\
        \texttt{SHR} $k, D$  & $D \leftarrow D >>_L k$ & Logical right shift \\
        \bottomrule
    \end{tabular}
\end{table}


As you notice, an operation is performed, and the result is written to the destination. 
Remember that all of them take \texttt{D} as the first argument:

\begin{verbatim}
add S, D   # D = D + S  (D comes first)
sub S, D   # D = D - S  (D comes first)
\end{verbatim}

\noindent
\textbf{Related Problem 5} (Updated from Practice Problem 3.10)

\begin{lstlisting}
short calc(short a, short b, short c) {

    short d = ______;
    short e = ______;
    short f = ______;
    short g = ______;
    return g;
    
}
\end{lstlisting}

The generated assembly code implementing these expressions is as follows:

\begin{lstlisting}
calc:
    movl    %esi, %eax
    orl     %esi, %edi
    sarw    $11, %di
    notl    %edi
    subl    %edi, %eax
    ret
\end{lstlisting}

\noindent
\textbf{Based on this assembly code, fill in the missing portions of the C code above.} \\
\\
\textit{Answer: }~\url{https://godbolt.org/z/aq9h3G8h8}
\subsection{LEA (Load effective address)}
First, let us remember the equation that is used to access certain parts of memory:
\noindent 
\\
\begin{tcolorbox}[colframe=black, colback=white, boxrule=1pt, title=Scaled Indexed Addressing Mode]
    \tcbtitle{Operands can denote immediate (constant) values, register values, or values 
    from memory. The scaling factor \( s \) must be either 1, 2, 4, or 8.}
    \begin{equation*}
        \text{Memory} \quad \textit{Imm}(r_b, r_i, s) = M[\textit{Imm} + R[r_b] + R[r_i] \cdot s]
    \end{equation*}
    \label{equation_mem}
\end{tcolorbox}
%
\noindent
%
In Related Problem 6, we will apply our equation to determine the memory address and 
then \textit{dereference} that address to access the value stored at it.
%
The LEA instruction uses the same equation for the calculation, but it 
\textbf{does not access memory} or \textbf{dereference the pointer}.

For example, imagine that register \%rax holds the value \texttt{0x10}. Additionally, 
the memory address \texttt{0x14} contains the value \texttt{0x20}. When we execute the 
instruction:
%
\[
\texttt{movq 4(\%rax), \%rsi}
\]
%
First, the \texttt{mov} instruction takes the value in \%rax, which is \texttt{0x10}. 
Then, based on the equation above, it adds \texttt{4} to this value: \texttt{0x10 + 4 = 0x14}. 
After that, it dereferences the memory address \texttt{0x14}, which holds the value \texttt{0x20}, and moves this value into the \%rsi register. \\
%
If we use the \texttt{LEA} instruction instead:
%
\[
\texttt{leaq 4(\%rax), \%rsi}
\]
%
The instruction computes \texttt{0x10 + 4 = 0x14} and moves this computed address 
directly into the \%rsi register without dereferencing the memory at \texttt{0x14}. 
As the name implies, \texttt{LEA} (Load Effective Address) loads the address, not 
the value stored at that address. \\
\\
%
\noindent\textbf{Related Problem 6} (Updated from Practice Problem 3.1) \\
Assume the following values are stored at the indicated memory addresses and registers:

\begin{table}[h!]
    \centering
    \small
    \renewcommand{\arraystretch}{1.2}
    \begin{tabular}{ll ll}
        \toprule
        \textbf{Address} & \textbf{Value} & \textbf{Register} & \textbf{Value} \\
        \midrule
        0x100 & 0xFF  & \%rax & 0x100 \\
        0x104 & 0xAB  & \%rcx & 0x1 \\
        0x108 & 0x13  & \%rdx & 0x3 \\
        0x10C & 0x11  & & \\
        \midrule
        \multicolumn{4}{l}{\textbf{Fill in the following values for the indicated 
        operands (Use Equation~\ref{equation_mem}):}} \\
        \midrule
        \textbf{Operand} & \textbf{Value (not used LEA)} & \textbf{Value (used LEA)} \\
        \midrule
        \%rax & \underline{\hspace{3cm}} & \underline{\hspace{3cm}} \\
        0x104 & \underline{\hspace{3cm}} & \underline{\hspace{3cm}} \\
        \$0x108 & \underline{\hspace{3cm}} & \underline{\hspace{3cm}} \\
        (\%rax) & \underline{\hspace{3cm}} & \underline{\hspace{3cm}} \\
        4(\%rax) & \underline{\hspace{3cm}} & \underline{\hspace{3cm}} \\
        9(\%rax,\%rdx) & \underline{\hspace{3cm}} & \underline{\hspace{3cm}} \\
        260(\%rcx,\%rdx) & \underline{\hspace{3cm}} & \underline{\hspace{3cm}} \\
        0xFC(,\%rcx,4) & \underline{\hspace{3cm}} & \underline{\hspace{3cm}} \\
        (\%rax,\%rdx,4) & \underline{\hspace{3cm}} & \underline{\hspace{3cm}} \\
        \bottomrule
    \end{tabular}
\end{table}
%
Compilers take advantage of \texttt{lea} to perform arithmetic operations. For 
example, Related Problem 7 demonstrates a case where the use of \texttt{lea} is 
unrelated to memory addresses but instead serves purely for arithmetic calculations: \\
\\
%
\noindent\textbf{Related Problem 7}
%
Consider the following code, in which we have omitted the expression being computed:
%
\begin{verbatim}
long calc(long a, long b, long c) {
    long result = ____________;
    return result;
}
\end{verbatim}
%
Compiling the actual function with \texttt{gcc} yields the following assembly code:
%
\begin{verbatim}
calc:
    leaq    (%rdi,%rdi,2), %rax
    leaq    (%rsi,%rax,4), %rax
    leaq    3(%rax,%rdx), %rax
    ret
\end{verbatim}
%
\textbf{Fill in the missing expression in the C code above.} \\
\\
\textit{Answer: }~\url{https://godbolt.org/z/rT5cYaY6z}
%
%% End of file