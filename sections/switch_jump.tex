\section{Switch case and Jump tables}
Compilers translate \texttt{switch} statements into \textbf{jump tables} when cases are dense. 
The best way is to look at an example. 
\\
For the given example below, instead of multiple comparisons, an index (\texttt{n - 100}) maps directly to an entry in \texttt{.L4}. For example, \texttt{case 104:} lacks a \texttt{break}, so it \textbf{falls through} to \texttt{case 106:}, meaning both use the same \texttt{.L3} label in assembly. The \textbf{default case} is \texttt{.L8}, where out-of-range values jump, setting \texttt{x = 0}. \textbf{Bias correction} is needed because \texttt{switch} cases may not start at 0; the compiler subtracts 100 (\texttt{subq \$100, \%rsi}) to normalize indexing, ensuring efficient lookups in the jump table; so the index may start from zero not from 100. \\
\\
Please carefully examine the following example, click the \textit{Compiler Explorer link }(\url{https://godbolt.org/z/EoPx6zooT}) and play with the source code to see how it compiles to a jump table. \\

\begin{verbatim}
void switch_case(long x, long n, long *dest)
{
    switch (n) {
        case 100: //  .L7
            x += 1;
            break;
        // Missing case 101: goes to default .L8
        case 102: // quad .L6
            x += 2;
            break;
        case 103: // .L5
            x += 3;
            break;
        case 104:
            /* Fall through  to case 106: which is .L3 */
        // Missing case 105: gos to default .L8
        case 106: // .L3
            x += 4;
            break;
        default:
            x = 0;
    }
    *dest = x;
}
\end{verbatim}
The code above compiles to:
\begin{verbatim}
 switch_case:
        subq    $100, %rsi    # Add bias so it starts from 0: 
        cmpq    $6, %rsi      # Max case number 106 now is 6;
                              # 6 = 106- 100 because of above bias
        ja      .L8           # If more than 6 jump to default case .L8
        jmp     *.L4(,%rsi,8) # Otherwise jump to the table to select
.L4:
        .quad   .L7 # Case 100 jumps to .L7
        .quad   .L8 # Case 101 is missing; jmp to default
        .quad   .L6 # Case 102 jumps to .L6
        .quad   .L5 # Case 103 jumps to .L5
        .quad   .L3 # Case 104 falls through to case 6
        .quad   .L8 # Case 105 is missing; jmp to default
        .quad   .L3 # Case 106 jumps to .L3 like case 104 did because of fall through
.L7:
        addq    $1, %rdi     # x+= 1;
        jmp     .L2
.L6:
        addq    $2, %rdi     # x+= 2;
        jmp     .L2
.L5:
        addq    $3, %rdi     # x+= 3;
        jmp     .L2
.L3:
        addq    $4, %rdi     # x+= 4;
.L2:
        movq    %rdi, (%rdx) # *dest = x;
        ret
.L8:
        movl    $0, %edi     # x = 0;
        jmp     .L2
\end{verbatim}

See the source code here: \url{https://godbolt.org/z/EoPx6zooT} \\
\\
\clearpage
\noindent\textbf{Related Problem 15} \\
The \texttt{switch\_case} example above specifies that any value less than zero must go to the default case. However, there is no explicit handling of negative values in the assembly code. 

Explain how the assembly code handles these cases.\\ 
\\
\noindent\textbf{Related Problem 16} (Updated from Practice Problem 3.31) \\
Complete the following C code based on the assembly code given.\\
The given C code
\begin{verbatim}
    void switcher(long a, long b, long c, long *dest)
{
    long val;
    switch(a) {
    case ___:
        c = ___;
        break;
    case ___:
        val = ___;
        break;
    case ___:
        /* Fall through */
    case ___:
        val = ___;
        break;
    case ___:
        val = ___;
        break;
    default:
        val = ___; 
    }
    *dest = val; 
}
\end{verbatim}
compiles to the following assembly code:
\begin{verbatim}
    switcher:
        cmpq    $7, %rdi
        ja      .L5
        jmp     *.L4(,%rdi,8)
.L4:
        .quad   .L7
        .quad   .L5
        .quad   .L3
        .quad   .L5
        .quad   .L6
        .quad   .L8
        .quad   .L5
        .quad   .L3
.L6:
        movq    %rdi, %rsi
        jmp     .L5
.L7:
        leaq    112(%rdx), %rsi
        jmp     .L5
.L3:
        addq    %rdx, %rsi
        salq    $2, %rsi
.L5:
        movq    %rsi, (%rcx)
        ret
.L8:
        movl    $0, %esi
        jmp     .L5
\end{verbatim}

\textit{Answer: }\url{https://godbolt.org/z/cnaGd5no3}
\clearpage