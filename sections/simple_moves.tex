\section{Simple moves}
\subsection{Move with different sizes}
\begin{table}[h]
    \centering
    \small
    \renewcommand{\arraystretch}{1.2}
    \begin{tabular}{l l c c}
        \toprule
        \textbf{C declaration} & \textbf{Intel data type} & \textbf{Assembly-code suffix} 
        & \textbf{Size (bytes)} \\
        \midrule
        char/unsigned char     & Byte             & b  & 1 \\
        short/unsigned short    & Word             & w  & 2 \\
        int/unsigned int      & Double word      & l  & 4 \\
        long/unsigned long     & Quad word        & q  & 8 \\
        char* (or any pointer)  & Quad word        & q  & 8 \\
        float    & Single precision & s  & 4 \\
        double   & Double precision & l  & 8 \\
        \bottomrule
    \end{tabular}
    \caption{Sizes of C data types in x86-64. With a 64-bit machine, pointers are 8 bytes long.}
    \label{tab:c_data_types}
\end{table}


\noindent\textbf{Related Problem 1} (Updated from Practice Problem 3.2)

\vspace{5pt}

For each of the following lines of assembly language, determine the appropriate instruction 
suffix based on the operands. (For example, \texttt{mov} can be rewritten as \texttt{movb}, 
\texttt{movw}, \texttt{movl}, or \texttt{movq}.) \\
%
\textbf{Hint:} Choose the suffixes that are compatible with the smaller-sized register. You 
can extract one byte from an 8-byte register, but you cannot extract 8 bytes from a 1-byte 
register.
%

\vspace{10pt}

\begin{lstlisting}[basicstyle=\ttfamily, frame=none]
mov__   %eax, (%rsp)        
mov__   (%rax), %dx         
mov__   $0xFF, %bl          
mov__   (%rsp,%rdx,4), %d1  
mov__   (%rdx), %rax        
mov__   %dx, (%rax)        
\end{lstlisting} 
%
\noindent\textbf{Related Problem 2} (Updated from Practice Problem 3.3)

\vspace{5pt}
%
Each of the following lines of code generates an error message when we invoke the assembler. 
Explain what is wrong with each line and write the corrected version.
%
\vspace{10pt}
%
\begin{lstlisting}[basicstyle=\ttfamily, frame=none]
movb  $0xF, (%ebx)    | Correct version: 
movl  %rax, (%rsp)    | Correct version:
movw  (%rax), 4(%rsp) | Correct version:
movb  %al, %sl        | Correct version:
movq  %rax, $0x123    | Correct version:
movl  %eax, %rdx      | Correct version:
movb  %si, 8(%rbp)    | Correct version:
\end{lstlisting}
%
\clearpage 
%
\subsection{Move with Extensions}
%
\begin{table}[t] 
    \centering
    \small
    \renewcommand{\arraystretch}{1.1}
    \begin{tabular}{l l l}
        \toprule
        \textbf{Instruction} & \textbf{Effect} & \textbf{Description} \\
        \midrule
        \multicolumn{3}{c}{\textbf{Sign-Extending Instructions}} \\
        \midrule
        \texttt{movs} $S, R$  & $R \leftarrow \text{SignExtend}(S)$ & Move with sign extension \\
        \texttt{movsbw} &  & Move sign-extended byte to word \\
        \texttt{movsbl} &  & Move sign-extended byte to double word \\
        \texttt{movswl} &  & Move sign-extended word to double word \\
        \texttt{movsbq} &  & Move sign-extended byte to quad word \\
        \texttt{movswq} &  & Move sign-extended word to quad word \\
        \texttt{movslq} &  & Move sign-extended double word to quad word \\
        \texttt{cltq}   & \%rax $\leftarrow$ SignExtend(\%eax) & Sign-extend \%eax to \%rax \\
        \midrule
        \midrule 
        \multicolumn{3}{c}{\textbf{Zero-Extending Instructions}} \\
        \midrule
        \texttt{movz} $S, R$  & $R \leftarrow \text{ZeroExtend}(S)$ & Move with zero extension \\
        \texttt{movzbw} &  & Move zero-extended byte to word \\
        \texttt{movzbl} &  & Move zero-extended byte to double word \\
        \texttt{movzwl} &  & Move zero-extended word to double word \\
        \texttt{movzbq} &  & Move zero-extended byte to quad word \\
        \texttt{movzwq} &  & Move zero-extended word to quad word \\
        \texttt{movzlq (not exist)} &  & \textbf{!!! Why do we not have this?} \\
        \bottomrule
    \end{tabular}
    \caption{Move with extensions. As you notice, we do not have \texttt{movzlq} 
    because it would be unnecessary based on the following conventions.}
\end{table}
%
\noindent\textbf{Register Updates Based on Data Movement Instructions}
%
\vspace{5pt}
%
As there are two conventions for data movement, the remaining bytes in the register 
are affected differently:
\begin{itemize}
    \item Instructions generating \textbf{1- or 2-byte} quantities leave the remaining 
    bytes unchanged.
    \item Instructions generating \textbf{4-byte} quantities set the upper 4 bytes of 
    the register to zero.
\end{itemize}
%
\vspace{10pt}
%
\begin{lstlisting}[basicstyle=\ttfamily, frame=none, numbers=left, numberstyle=\color{blue}]
movabsq $0x0011223344556677, %rax    %rax = 0011223344556677
movb    $-1, %al                     %rax = 00112233445566FF
movw    $-1, %ax                     %rax = 001122334455FFFF
movl    $-1, %eax                    %rax = 00000000FFFFFFFF
movq    $-1, %rax                    %rax = FFFFFFFFFFFFFFFF
\end{lstlisting}
%
That is why \texttt{movl \$-1, \%eax} moves the value to \%eax and automatically 
\texttt{zero-extends} (zeros out the upper 4 bytes) due to the conventions mentioned 
above. You may also see \texttt{movl \%eax, \%eax}, which is a simple trick to zero 
out the upper bytes.\\
%
\textbf{FWIW:} Do not worry about \texttt{movabsq \$0x0011223344556677, \%rax}; it is 
simply used to move very large numbers. When a 64-bit number can be represented as a 
32-bit value, the compiler uses \texttt{mov}. If it cannot, then it uses \texttt{movabsq}. 
%
\texttt{movabsq} can have a register as its destination, but it cannot write directly to memory. 
See an example here:
~\url{https://godbolt.org/z/bsvjGsf3G} \\
\\
Most of the time you will see \texttt{xorl \%eax, \%eax} that is used to zero out 
the \texttt{\%rax}. Based on the Table~\ref{truth-table} xoring the same value with 
itself will result in zero. Since zeroing \texttt{\%eax} also zeros out the upper 4 
bytes, compiler chooses \texttt{xorl \%eax, \%eax} which is cheaper than \texttt{movq \$0, \%rax} \\
\\
%
\noindent\textbf{Related Problem 3} (Updated from Practice Problem 3.5) \\
%
\vspace{5pt}
%
You are given the following information. A function with the prototype:
%
\begin{lstlisting}[basicstyle=\ttfamily, frame=none]
void decode(long *xp, long *yp, long *zp);
\end{lstlisting}

is compiled into assembly code, yielding the following:
%
\begin{lstlisting}[basicstyle=\ttfamily, frame=none]
decode:
    movq    (%rdi), %r8
    movq    (%rsi), %rcx
    movq    (%rdx), %rax
    movq    %r8, (%rsi)
    movq    %rcx, (%rdx)
    movq    %rax, (%rdi)
    ret
\end{lstlisting}
%
\vspace{5pt}
%
Parameters \texttt{xp}, \texttt{yp}, and \texttt{zp} are stored in registers 
\_\_\_\_, \_\_\_\_, and \_\_\_\_, respectively (Remember calling conventions 
from~\S\ref{sec: how-args-passed}.)
%
\vspace{5pt}
%
Write C code for \texttt{decode} that will have an effect equivalent to the 
assembly code shown.\\
\begin{lstlisting}[basicstyle=\ttfamily, frame=none]
// Complete the function below
void decode(long *xp, long *yp, long *zp) {







}
\end{lstlisting}
\textit{Answer: }~\url{https://godbolt.org/z/9G4696P6j} 
%
\clearpage
%
\noindent\textbf{Related Problem 4} (Updated from Practice Problem 3.4)
%
\begin{lstlisting}
void cast(TYPE_1 *sp, TYPE_2 *dp) {
    *dp = (TYPE_2) *sp;
}
\end{lstlisting}
%
You are given this function that casts a source value (\texttt{*sp}) and writes it 
to a destination (\texttt{*dp}). Write the instructions necessary to perform the same 
operation with different \texttt{TYPE\_1} and \texttt{TYPE\_2}. An example is provided for you.
%
\textbf{Remember:}
\begin{itemize}
    \item \textbf{You cannot move from memory to memory.}
    \item When performing a cast that involves both a \textbf{size change} and a 
    \textbf{change of signedness} in C, the operation should change the size first.
\end{itemize}
%
\vspace{10pt}
%
\begin{table}[h]
    \centering
    \small
    \renewcommand{\arraystretch}{1.4} % Adjusts row height for two lines
    \begin{tabular}{l l p{5cm}} % p{5cm} allows two-line content
        \toprule
        \textbf{TYPE\_1} & \textbf{TYPE\_2} & \textbf{Instruction} \\
        \midrule
        long & long & \texttt{movq (\%rdi), \%rax} \newline \texttt{movq \%rax, (\%rsi)} \\
        char & int & \underline{\hspace{5cm}} \newline \underline{\hspace{5cm}} \\
        char & unsigned & \underline{\hspace{5cm}} \newline \underline{\hspace{5cm}} \\
        unsigned char & long & \underline{\hspace{5cm}} \newline \underline{\hspace{5cm}} \\
        int & char & \underline{\hspace{5cm}} \newline \underline{\hspace{5cm}} \\
        unsigned & unsigned char & \underline{\hspace{5cm}} \newline \underline{\hspace{5cm}} \\
        char & short & \underline{\hspace{5cm}} \newline \underline{\hspace{5cm}} \\
        \bottomrule
    \end{tabular}
\end{table}
\textit{Answer:} \url{https://godbolt.org/z/ePox4Y7sv}
\clearpage
%
%% End of file