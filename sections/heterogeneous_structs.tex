\section{Heterogeneous data structures}
\subsection{Struct}
Different ways to declare the structure: \\
\\
\noindent
\begin{minipage}{0.45\textwidth}
\begin{verbatim}
// Standard struct declaration
struct rec {
    int i;
    int j;
    int a[2];
    int *p;
};
\end{verbatim}
\end{minipage}
\hfill
\begin{minipage}{0.45\textwidth}
\begin{verbatim}
// Using typedef with struct
typedef struct rec_t {
    int i;
    int j;
    int a[2];
    int *p;
} rec_t;
\end{verbatim}
\end{minipage}
%
\vspace{1em}
%
\noindent
\begin{minipage}{0.45\textwidth}
\begin{verbatim}
// Anonymous struct with typedef
typedef struct {
    int i;
    int j;
    int a[2];
    int *p;
} rec_anonymous;
\end{verbatim}
\end{minipage}
%
\vspace{1em}
%
This structure contains four fields: two 4-byte values of type \texttt{int}, 
a two-element array of type \texttt{int}, and an 8-byte integer pointer, 
giving a total of 24 bytes:
%
\begin{table}[h]
    \centering
    \renewcommand{\arraystretch}{1.5}
    \begin{tabular}{c c c c c c}
        \toprule
        \textbf{Offset} & 0 & 4 & 8 & 16 & 24 \\
        \midrule
        \textbf{Contents} & \texttt{i} & \texttt{j} & \texttt{a[0]} & \texttt{a[1]} & \texttt{p} \\
        \bottomrule
    \end{tabular}
\end{table}
\subsection{Padding}
%
Fields of a \texttt{struct} in C are stored in the order in which they are 
declared, and the C standard specifies that this order cannot be rearranged 
in memory. Additionally, many computer systems impose alignment restrictions 
on \textbf{primitive} data types, requiring that certain objects be placed at 
addresses that are multiples of a specific value \( K \) (typically 2, 4, or 8). 
These alignment constraints simplify the design of hardware that interfaces 
between the processor and the memory system. To satisfy these alignment requirements, 
padding may be inserted between \texttt{struct} members. \\
\\
%
\clearpage
%
The easiest way to determine padding follows two rules:
%
\begin{enumerate}
    \item Starting from address zero check if the offset where a variable 
    starts is divisible by the size of its type:
    \begin{itemize}
        \item If yes, place it without adding any padding.
        \item If no, round up to the nearest multiple by adding padding.
    \end{itemize}
    
    \item After placing all fields, check the total size of the struct:
    \begin{itemize}
        \item Is the total size divisible by the largest data type's size?
        \item If not, round it up to the nearest multiple by adding padding.
    \end{itemize}
\end{enumerate}

Let us consider the following example:

\noindent\begin{minipage}{0.45\textwidth}
\begin{lstlisting}
typedef struct {
    short a; 
    double b;
    int c;
} my_struct, *p_my_struct;
\end{lstlisting}
\end{minipage}
%
We have declared \texttt{my\_struct} and \texttt{p\_my\_struct}, which is a 
pointer to \texttt{my\_struct}. The size of \texttt{p\_my\_struct} is always 
8 bytes, as it is a pointer. This is the same as declaring a variable of 
type \texttt{my\_struct*}. Writing \texttt{p\_my\_struct a;} and \texttt{my\_struct* a;} 
are equivalent.
%
Now, let's determine the memory layout of \texttt{my\_struct} using Rule 1.
%
\begin{enumerate}
    \item \textbf{Starting at address 0:}  
    Is 0 divisible by the size of \texttt{short} (2 bytes)?  
    Yes, because \( 0 \mod 2 = 0 \).  
    We place \texttt{a} at address 0, occupying 2 bytes.

    \item \textbf{Next, adding \texttt{double b}:}  
    The next available address is 2.  
    Is 2 divisible by the size of \texttt{double} (8 bytes)?  
    No, because \( 2 \mod 8 \neq 0 \).  
    To align \texttt{b} properly, we add 6 bytes of padding.  
    Now, \texttt{b} starts at address 8 and occupies 8 bytes.  
    The next available address is \( 8 + 8 = 16 \).

    \item \textbf{Adding \texttt{int c}:}  
    Is 16 divisible by the size of \texttt{int} (4 bytes)?  
    Yes, because \( 16 \mod 4 = 0 \).  
    We place \texttt{c} at address 16, occupying 4 bytes.  
    The total struct size at this point is \( 16 + 4 = 20 \) bytes.

\end{enumerate}

Now, applying Rule 2:

\begin{itemize}
    \item The largest data type in the struct is \texttt{double} (8 bytes).
    \item Is the total struct size (20 bytes) divisible by 8?  
    No, because \( 20 \mod 8 \neq 0 \).
    \item To satisfy alignment requirements, we add 4 bytes of padding at the 
    end to make the total size 24, which is divisible by 8.
\end{itemize}

\textbf{Final Struct Layout:}

\begin{verbatim}
// | a | padding(6) | b       | c  | padding(4) |
// ^0  ^2          ^8        ^16  ^20          ^24
\end{verbatim}

Thus, the final size of \texttt{my\_struct} is \textbf{24 bytes}. \\
See this code and see the output for the same struct: \\

~\url{https://godbolt.org/z/anrTEceE1} \\

Please carefully analyze the following examples and understand. They also 
include nested structs. You can run the examples below here:~\url{https://godbolt.org/z/YzohaaoEz} \\
\\
\begin{verbatim}
    typedef struct {
        short a; 
        int b;
        int* c;
        short* d;
    } p1;
    \end{verbatim}
    
    \begin{verbatim}
    // | a | padding(2) | b | c | d |
    //   ^2             ^4  ^8   ^16 
    // size = 16 + 8 (short* d) = 24
    \end{verbatim}
    
    \vspace{1em}
    
    \begin{verbatim}
    typedef struct {
        int a[2];
        char b[8];
        short c[4];
        long* d;
    } p2;
    \end{verbatim}
    
    \begin{verbatim}
    // | a[0] | a[1] | b[0] | b[1] | ... | b[7] | c[0] | ... | c[3] | d
    // ^0     ^4     ^8     ^9     ^10          ^16     ^18         ^24
    // size = 24 + 8 (long* d) = 32
    \end{verbatim}
    
    \vspace{1em}
    
    \begin{verbatim}
    typedef struct {
        long a[2];
        int* b[2];
    } p3;
    \end{verbatim}
    
    \begin{verbatim}
    // | a[0] | a[1] | b[0] | b[1]
    // ^0     ^8     ^16    ^24
    // size = 24 + 8(b[1]) = 32
    \end{verbatim}
    
    \vspace{1em}
    
    \begin{verbatim}
    typedef struct {
        char a[16];
        char* b[2];
    } p4, *p_p4;
    \end{verbatim}
    
    \begin{verbatim}
    // | a[0] | a[1] | .... | a[15] | b[0] | b[1] 
    // ^0     ^1     ^2     ^15     ^16    ^24
    // size = 24 + 8(b[1]) = 32
    \end{verbatim}
    
    \vspace{1em}
    
    \begin{verbatim}
    typedef struct {
        p4 a[4];
        p1 b;
    } p5;
    \end{verbatim}
    
    \begin{verbatim}
    // | a[0] | a[1] | a[2] | a[3] | b
    // ^0     ^32    ^64    ^96     ^128 
    // size = 128 + 24(b) = 152
    \end{verbatim}
    
    \vspace{1em}
    
    \begin{verbatim}
    typedef struct {
        int a;
        int b;
        long c;
        char d;
        char e;
    } p6;
    \end{verbatim}
    
    \begin{verbatim}
    // | a | b | c | d | e |
    // ^0  ^4  ^8  ^16  ^17
    // size = 17 + 7 (!!! The total structure size must be a multiple
    // of the largest alignment)
    // In our case long is the largest member. 
    // Total size must be 24 (rounded up to the nearest multiple of 8).
    \end{verbatim}
\noindent\textbf{Note:} \\
%
As you may notice, alignment and our struct padding rules are based on 
primitive data types. This is why, in the case of \texttt{p5}, we did not 
apply our rule that checks whether 128 (the size of \texttt{p4 a[4]}) is 
divisible by 24 (the size of \texttt{p1 b}). Instead, we applied padding 
based on the alignment requirements of the individual fields.
%
Additionally, we did not apply our second rule, which checks whether the 
total size is divisible by the size of the largest field. In this case, 
the largest field is \texttt{p4}, which has a size of 32 bytes, but the 
total struct size of 152 is not divisible by 32. Again, this is because 
we align based on primitive data types.
%
Since structs do not exist at the assembly level, you can think of nested 
structs as being "unrolled" into memory, where each nested struct is treated 
as a contiguous block before aligning the next field.