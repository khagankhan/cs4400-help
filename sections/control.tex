\section{Control}
\subsection{Setting flags}
In addition to the integer registers, the CPU maintains a set of single-bit 
\textit{condition code} registers describing attributes of the most recent arithmetic 
or logical operation.
The \texttt{leaq} instruction does not alter any condition codes, since it 
is intended to be used in address computations. Otherwise, all of the instructions 
listed in Table~\ref{basic-arith} cause the condition codes to be set.
In addition to the instructions in the Table~\ref{basic-arith} there are 
two instruction classes (having 8-, 16-, 32-, and 64-bit forms) that set 
condition codes without altering any other registers: \texttt{cmp} and \texttt{test}:
\begin{table}[h]
    \centering
    \begin{tabular}{lll}
        \toprule
        \textbf{Instruction} & \textbf{Based on} & \textbf{Description} \\
        \midrule
        \texttt{CMP}  & $S_1, S_2$ \quad $S_2 - S_1$ & Compare \\
        \quad \texttt{cmpb} & & Compare byte \\
        \quad \texttt{cmpw} & & Compare word \\
        \quad \texttt{cmpl} & & Compare double word \\
        \quad \texttt{cmpq} & & Compare quad word \\
        \midrule
        \texttt{TEST} & $S_1, S_2$ \quad $S_1 \& S_2$ & Test \\
        \quad \texttt{testb} & & Test byte \\
        \quad \texttt{testw} & & Test word \\
        \quad \texttt{testl} & & Test double word \\
        \quad \texttt{testq} & & Test quad word \\
        \bottomrule
    \end{tabular}
    \caption{Comparison and test instructions. These instructions set the 
    condition codes without updating any other registers.}
    \label{tab:cmp_test_instructions}
\end{table}
%
As you notice, \texttt{cmp} and \texttt{test} use \textit{subtraction} 
and \textit{and}, respectively. But what makes them different from the 
corresponding \textit{sub} and \textit{and} instructions in Table~\ref{basic-arith} 
is that \texttt{cmp} and \texttt{test} \textbf{DO NOT UPDATE THE VALUE IN THE DESTINATION 
REGISTER}; they just set the flags.\\
%
See the following example.
\\
\begin{tabular}{l l}
\texttt{movq \$5, \%rax}   & \texttt{movq \$5, \%rax}   \\
\texttt{movq \$2, \%rbx}   & \texttt{movq \$2, \%rbx}   \\
\texttt{subq \%rax, \%rbx} & \texttt{cmpq \%rax, \%rbx} \\ 
\texttt{\# \%rbx = \%rbx - \%rax} & \texttt{\# \%rbx - \%rax, only sets flags} \\ 
                                & \texttt{\# not assigns val to \%rbx} \\ 
\\
\textbf{After Execution (subq):}  & \textbf{After Execution (cmpq):}  \\
\texttt{\# \%rax = 5}      & \texttt{\# \%rax = 5}      \\
\texttt{\# \%rbx = -3}     & \texttt{\# \%rbx = 2  (unchanged)} \\
\texttt{\# SF = 1 (negative result)} & \texttt{\# SF = 1 (negative result)}
\end{tabular}
\\
\\
This demonstrates that while \texttt{subq} updates the destination 
register (\%rbx), \texttt{cmpq} performs the same subtraction but 
only modifies the processor flags without changing the register contents. 
The same idea holds true for the \texttt{test} instruction, too.


For example, suppose the \texttt{cmpl a b} instruction to perform that 
does \texttt{b - a}, where variables \texttt{a}, \texttt{b} are integers. 
Then the condition codes would be set according to the following C-like expressions:

\begin{Verbatim}[frame=single]
CF  (unsigned) b < (unsigned) a                Unsigned overflow
ZF  (b == a)                                   Zero
SF  ((b - a) < 0)                              Negative
OF  (a < 0 && b > 0 && (b - a) < 0) || 
    (a > 0 && b < 0 && (b - a) > 0)            Signed overflow
\end{Verbatim}
\noindent\textbf{\textcolor{red}{Note:}} \\
Please consider the above as a "C-like expression" rather than actual C code. 
The authors of the CSAPP book~\cite{bryant2016csapp} may have mistakenly applied a similar logic to C code. For more 
details, please refer to \texttt{Lec04.pdf}, slide number 83, and Solution 2.30
from the book of the mentioned edition in the Acknowledgment~\S\ref{sec:ack}. \\
\\
\noindent\textbf{Related Problem 8} \\
It is easier to see a concept in smaller numbers, so we first consider 4-bit 
values for condition codes. The concept is the same, since 4-bit range is 
smaller it makes things easier. \\

For example: 
\begin{itemize}
    \item \textbf{For 4-bit integers:}
    \begin{itemize}
        \item Unsigned values range from $0$ to $2^4 - 1 = 15$.
        \item Signed values range from $-2^3 = -8$ to $2^3 - 1 = 7$.
    \end{itemize}
    \item The same logic applies to 32-bit integers, except that:
    \begin{itemize}
        \item Unsigned values range from $0$ to $2^{32} - 1$.
        \item Signed values range from $-2^{31}$ to $2^{31} - 1$.
    \end{itemize}
\end{itemize}

\begin{table}[h]
    \centering
    \renewcommand{\arraystretch}{1.2}
    \begin{tabular}{|c|c|c|}
        \hline
        \textbf{Binary} & \textbf{Unsigned Value (0 to 15)} & \textbf{Signed Value (-8 to 7)} \\
        \hline
        0000 & 0  & 0  \\
        0001 & 1  & 1  \\
        0010 & 2  & 2  \\
        0011 & 3  & 3  \\
        0100 & 4  & 4  \\
        0101 & 5  & 5  \\
        0110 & 6  & 6  \\
        0111 & 7  & 7  \\
        1000 & 8  & -8 \\
        1001 & 9  & -7 \\
        1010 & 10 & -6 \\
        1011 & 11 & -5 \\
        1100 & 12 & -4 \\
        1101 & 13 & -3 \\
        1110 & 14 & -2 \\
        1111 & 15 & -1 \\
        \hline
    \end{tabular}
    \caption{4-bit Unsigned and Signed Integer Representation}
    \label{four-bit-int}
\end{table}


Given two \textbf{4-bit numbers} (Use Table~\ref{four-bit-int} as a reference) 
$a$ and $b$ in two’s complement representation:

\begin{enumerate}
    \item $a = 5$, $b = 3$
    \item $a = 7$, $b = -6$
    \item $a = -8$, $b = 7$
    \item $a = -3$, $b = -5$
    \item $a = -8$, $b = -8$
\end{enumerate}

For each case, compute the result of \texttt{cmp a b} and determine 
the values of the following flags: \textbf{CF, ZF, SF, OF} \\
\\
%
\clearpage
%
\noindent\textbf{Related Problem 9}  \\
%
From \textbf{Related Problem 8}, for each case, compute the result of 
\texttt{add a b} (Use Table~\ref{four-bit-int} as a reference) and determine 
the values of the same flags (In the \textbf{Related Problem 8} you used the 
conditions for \texttt{cmp a b} now you are supposed to work with \texttt{add a b}.) \\ 
First, try to write C-like expressions for \texttt{add a b} for the given flags: \\
%
\begin{Verbatim}[frame=single]
CF                                             Unsigned overflow
ZF                                             Zero
SF                                             Negative
OF   
                                               Signed overflow
\end{Verbatim}

Then, determine the value of the flags after the operation \texttt{add a, b} 
for the following (Numbers are still \textbf{4-bit numbers}): \\
\begin{enumerate}
    \item $a = 5$, $b = 3$
    \item $a = 7$, $b = -6$
    \item $a = -8$, $b = 7$
    \item $a = -3$, $b = -5$
    \item $a = -8$, $b = -8$
\end{enumerate}

\noindent\textbf{Note:}  

As discussed in the lecture, Prof. Dr. Regehr suggested a simpler approach to handling flags: \\
\\
\textit{Cast register values to a signed 64-bit integer type, perform the arithmetic 
operation, and then compare the result against \texttt{INT\_MIN} and \texttt{INT\_MAX}. 
However, ensure that when casting to the larger type, the conversion correctly 
sign-extends rather than zero-extends.} \\
\\
This approach may be easier to implement. You can apply the same concept to the 
above Related Problems (Related Problem 8 and Related Problem 9). Since the numbers 
in those problems are 4-bit values, you can extend them to 8-bit and perform the 
comparison accordingly.
%
\clearpage
%
\subsection{Usage of Flags}
%
Now that we understand how flags are set, what do these flags actually signify? Their usage can be categorized into three main operations:
%
\begin{enumerate}
    \item Setting a single byte to \texttt{0} or \texttt{1} based on a specific 
    combination of condition codes (\texttt{SET} instructions are used).
    \item Conditionally jumping to another part of the program (\texttt{JMP} 
    instructions are used).
    \item Conditionally transferring data (\texttt{CMOV} instructions are used.)
\end{enumerate}
%
Please see Table~\ref{set-and-jmp} for the conditions of the operations 
listed above. As you can see, they use the same suffixes, which change 
based on whether the values are signed or unsigned. The suffixes \textbf{a} 
(above) and \textbf{b} (below) are used for unsigned values, while the suffixes 
\textbf{g} (greater) and \textbf{l} (less) are used for signed values. 
These suffixes determine how different flag combinations affect the result. 
%
Additionally, please familiarize yourself with the truth tables for XOR, 
OR, NOT, and AND operations, as shown in Table~\ref{truth-table}, if you 
are not already familiar with them.
%
\begin{table}[h]
    \centering
    \small
    \renewcommand{\arraystretch}{1.2}
    \begin{tabular}{c c | c c c c c}
        \toprule
        $A$ & $B$ & $A \& B$ & $A | B$ & $A \texttt{\string^} B$ & $\sim A$ & $\sim B$ \\
        \midrule
        0 & 0 & 0 & 0 & 0 & 1 & 1 \\
        0 & 1 & 0 & 1 & 1 & 1 & 0 \\
        1 & 0 & 0 & 1 & 1 & 0 & 1 \\
        1 & 1 & 1 & 1 & 0 & 0 & 0 \\
        \bottomrule
    \end{tabular}
    \caption{Truth table for AND ( $\&$ ), OR ( $|$ ), XOR ( \texttt{\string^} ), and NOT ( $\sim$ ).}
    \label{truth-table}
\end{table}

\clearpage
\begin{table}[h]
    \centering
    \small
    \renewcommand{\arraystretch}{1.2}
    \begin{tabular}{l l l l}
        \toprule
        \textbf{Instruction} & \textbf{Synonym} & \textbf{Effect} & \textbf{Condition} \\
        \midrule
        \multicolumn{4}{c}{\textbf{Set Instructions}} \\
        \midrule
        \texttt{sete} $D$  & \texttt{setz} $D$  & $D \leftarrow 
        ZF$ & Equal / zero \\
        \texttt{setne} $D$  & \texttt{setnz} $D$  & $D \leftarrow 
        \sim ZF$ & Not equal / not zero \\
        \texttt{sets} $D$  & – & $D \leftarrow SF$ & Negative \\
        \texttt{setns} $D$  & – & $D \leftarrow \sim SF$ & Nonnegative \\
        \texttt{setg} $D$  & \texttt{setnle} $D$  & $D \leftarrow 
        \sim (SF \texttt{\string^} OF) \& \sim ZF$ & Greater (signed $>$) \\
        \texttt{setge} $D$  & \texttt{setnl} $D$  & $D \leftarrow 
        \sim (SF \texttt{\string^} OF)$ & Greater or equal (signed $\geq$) \\
        \texttt{setl} $D$  & \texttt{setnge} $D$  & $D \leftarrow 
        SF \texttt{\string^} OF$ & Less (signed $<$) \\
        \texttt{setle} $D$  & \texttt{setng} $D$  & $D \leftarrow 
        (SF \texttt{\string^} OF) | ZF$ & Less or equal (signed $\leq$) \\
        \texttt{seta} $D$  & \texttt{setnbe} $D$  & $D \leftarrow 
        \sim CF \& \sim ZF$ & Above (unsigned $>$) \\
        \texttt{setae} $D$  & \texttt{setnb} $D$  & $D \leftarrow 
        \sim CF$ & Above or equal (unsigned $\geq$) \\
        \texttt{setb} $D$  & \texttt{setnae} $D$  & $D \leftarrow 
        CF$ & Below (unsigned $<$) \\
        \texttt{setbe} $D$  & \texttt{setna} $D$  & $D \leftarrow 
        CF | ZF$ & Below or equal (unsigned $\leq$) \\
        \midrule
        \multicolumn{4}{c}{\textbf{Jump Instructions}} \\
        \midrule
                \texttt{jmp} \textit{Label}  & – & Unconditional jump & – \\
        \texttt{je} \textit{Label}  & \texttt{jz} \textit{Label} & Jump if 
        $ZF = 1$ & Equal / zero \\
        \texttt{jne} \textit{Label}  & \texttt{jnz} \textit{Label} & Jump if 
        $ZF = 0$ & Not equal / not zero \\
        \texttt{js} \textit{Label}  & – & Jump if 
        $SF = 1$ & Negative \\
        \texttt{jns} \textit{Label}  & – & Jump if 
        $SF = 0$ & Nonnegative \\
        \texttt{jg} \textit{Label}  & \texttt{jnle} \textit{Label} & Jump if 
        $\sim (SF \texttt{\string^} OF) \& \sim ZF$ & Greater (signed $>$) \\
        \texttt{jge} \textit{Label}  & \texttt{jnl} \textit{Label} & Jump if 
        $\sim (SF \texttt{\string^} OF)$ & Greater or equal (signed $\geq$) \\
        \texttt{jl} \textit{Label}  & \texttt{jnge} \textit{Label} & Jump if 
        $SF \texttt{\string^} OF$ & Less (signed $<$) \\
        \texttt{jle} \textit{Label}  & \texttt{jng} \textit{Label} & Jump if 
        $(SF \texttt{\string^} OF) | ZF$ & Less or equal (signed $\leq$) \\
        \texttt{ja} \textit{Label}  & \texttt{jnbe} \textit{Label} & Jump if 
        $\sim CF \& \sim ZF$ & Above (unsigned $>$) \\
        \texttt{jae} \textit{Label}  & \texttt{jnb} \textit{Label} & Jump if 
        $\sim CF$ & Above or equal (unsigned $\geq$) \\
        \texttt{jb} \textit{Label}  & \texttt{jnae} \textit{Label} & Jump if 
        $CF$ & Below (unsigned $<$) \\
        \texttt{jbe} \textit{Label}  & \texttt{jna} \textit{Label} & Jump if 
        $CF | ZF$ & Below or equal (unsigned $\leq$) \\
        \midrule
        \multicolumn{4}{c}{\textbf{Cmove Instructions}} \\
        \midrule
        \texttt{cmove} $S, R$  & \texttt{cmovz} & Move if $ZF = 1$ 
        & Equal / zero \\
        \texttt{cmovne} $S, R$  & \texttt{cmovnz} & Move if $ZF = 0$ 
        & Not equal / not zero \\
        \texttt{cmovs} $S, R$  & – & Move if $SF = 1$ 
        & Negative \\
        \texttt{cmovns} $S, R$  & – & Move if $SF = 0$ 
        & Nonnegative \\
        \texttt{cmovg} $S, R$  & \texttt{cmovnle} & Move if $\sim (SF \texttt{\string^} OF) \& \sim ZF$ 
        & Greater (signed $>$) \\
        \texttt{cmovge} $S, R$  & \texttt{cmovnl} & Move if $\sim (SF \texttt{\string^} OF)$ 
        & Greater or equal (signed $\geq$) \\
        \texttt{cmovl} $S, R$  & \texttt{cmovnge} & Move if $SF \texttt{\string^} OF$ & Less (signed $<$) \\
        \texttt{cmovle} $S, R$  & \texttt{cmovng} & Move if $(SF \texttt{\string^} OF) | ZF$ 
        & Less or equal (signed $\leq$) \\
        \texttt{cmova} $S, R$  & \texttt{cmovnbe} & Move if $\sim CF \& \sim ZF$ 
        & Above (unsigned $>$) \\
        \texttt{cmovae} $S, R$  & \texttt{cmovnb} & Move if $\sim CF$ 
        & Above or equal (unsigned $\geq$) \\
        \texttt{cmovb} $S, R$  & \texttt{cmovnae} & Move if $CF$ 
        & Below (unsigned $<$) \\
        \texttt{cmovbe} $S, R$  & \texttt{cmovna} & Move if $CF | ZF$ 
        & Below or equal (unsigned $\leq$) \\
        \bottomrule
    \end{tabular}
    \caption{Conditional set, move, and jump instructions in x86-64. These instructions 
    depend on condition flags set by previous arithmetic or logical operations.}
    \label{set-and-jmp}
\end{table}

\clearpage
\noindent Time for the problems: \\
\\
\noindent\textbf{Related Problem 10}  \\
Given the C code:
\begin{verbatim}
int comp(TYPE a, TYPE b) {
    return a COMP b;
}
\end{verbatim}
The code above shows a general comparison between arguments a and b
where TYPE  is the data type of the arguments.

For each of the following instruction sequences, determine which data 
types TYPE and which comparisons COMP could cause the compiler to 
generate this code. (There can be multiple correct answers; you should 
list them all.)
\noindent Instruction Sequence:
\begin{verbatim}
cmpl    %esi, %edi
setl    %al
movzbl  %al, %eax
ret
----------------------------------
cmpw    %si, %di
setge   %al
movzbl  %al, %eax
ret
----------------------------------
cmpb    %dil, %sil
setnb   %al
movzbl  %al, %eax
ret
----------------------------------
cmpq    %rsi, %rdi
setne   %al
movzbl  %al, %eax
ret
\end{verbatim}

\noindent\textit{Answer: }~\url{https://godbolt.org/z/ndrn6bqWY} \\
\\
\noindent\textbf{Related Problem 11}  \\
Given the C code:
\begin{verbatim}
int test(TYPE a) {
    return a COMP 0;
}
\end{verbatim}
The code above shows a general comparison between arguments a and b where TYPE is 
the data type of the arguments.

\noindent For each of the following instruction sequences, determine which data 
types TYPE and which comparisons COMP could cause the compiler to 
generate this code. (There can be multiple correct answers; you should 
list them all.)

\noindent Instruction Sequence:
\begin{verbatim}
testq   %rdi, %rdi
setle   %al
movzbl  %al, %eax
ret
----------------------------------
testw   %di, %di
sete    %al
movzbl  %al, %eax
ret
\end{verbatim}

\noindent\textit{Answer: }~\url{https://godbolt.org/z/5s5ffGPz5} \\
\\
\noindent\textbf{Related Problem 12} (Updated from Practice Problem 3.18) \\

Complete the C code below based on the assembly code provided after the C code.

\begin{verbatim}
long test(long x, long y, long z) {
    long val = __________;
    if (__________) {
        if (__________)
            val = __________;
        else
            val = __________;
    } else if (__________)
        val = __________;
    return val;
}
\end{verbatim}
GCC generates the following assembly code. Based on this assembly code complete 
the C code above.
\begin{verbatim}
test:
        leaq    (%rdx,%rsi), %rax
        movq    %rax, %rcx
        subq    %rdi, %rcx
        cmpq    $5, %rdx
        jle     .L2
        cmpq    $2, %rsi
        jle     .L3
        movq    %rdi, %rax
        subq    %rdx, %rax
        ret
.L3:
        movq    %rsi, %rax
        imulq   %rdi, %rax
        ret
.L2:
        cmpq    $2, %rdx
        jle     .L1
        movq    %rcx, %rax
.L1:
        ret
\end{verbatim}

\noindent\textit{Answer: }~\url{https://godbolt.org/z/TWezeEM6d} \\
\\
As we know, \texttt{jmp} instructions are used for loops since there are no 
explicit loop instructions at the assembly level:

\begin{enumerate}
    \item Compare the condition (check loop condition).
    \item If the condition is not met, jump back to the loop body.
    \item If the condition is met, do not jump back and execute the instructions 
    after the jump (execute code after the loop body).
\end{enumerate}


\noindent\textbf{Related Problem 13} \\
\begin{verbatim}
For C code having the general form:

long loop(long a, long b)
{
    long result = __________;
    while (__________) {
        result = __________;
        b = __________;
    }
    return result;
}
\end{verbatim}

\begin{verbatim}
loop:
        leaq    (%rsi,%rsi,2), %rax
        addq    %rdi, %rax
        jmp     .L2
.L3:
        addq    %rdi, %rax
        addq    $1, %rsi
.L2:
        cmpq    $10, %rsi
        jg      .L3
        ret
\end{verbatim} 

\noindent\textit{Answer: }~\url{https://godbolt.org/z/q7c1e45Mb} \\
\\
\noindent\textbf{Related Problem 14} \\
For the following C code
\begin{verbatim}
    long modify_long(long a) {
    if (a > 6 || a < 0) {
            a = a + 3;
    } else {
        a = a + 5;
    }
    return a;
}
\end{verbatim}
GCC generates the following assembly code with \texttt{-Og} flag:
\begin{verbatim}
    modify_long:
        cmpq    $6, %rdi
        jbe     .L2
        leaq    3(%rdi), %rax
        ret
.L2:
        leaq    5(%rdi), %rax
        ret
\end{verbatim}

You can see it here: \textit{https://godbolt.org/z/zdqTv8YWK} \\
\\
As you can see, after \texttt{cmpq \$6, \%rdi} compiler uses \textit{jbe    .L2} 
instruction.  However, as we know ftom Table~\ref{set-and-jmp} that \texttt{jbe} instruction 
is used for unsigned values while in our C code the type of \texttt{a} is \texttt{long} which 
is signed. \\
\\
Explain why compiler uses \texttt{jbe} even if the value is signed. \\
\\
\textbf{Hint:} Everything is just bits for the computer. See an example bit 
patterns for signed and unsigned values in the Table~\ref{four-bit-int}. \\
\\
\noindent\textbf{Related Problem 15} \\
For the C code below:
\begin{verbatim}
    long cmove(long a, long b) {
        if (_) {
            return _____;
        } else {
            return _____;
        }
}
\end{verbatim}
The compiler generates the following assembly code: 
\begin{verbatim}
    cmove:
        leaq    (%rdi,%rsi), %rax
        addq    $3, %rsi
        testq   %rdi, %rdi
        cmove   %rsi, %rax
        ret
\end{verbatim}
Based on the assembly code, complete the given C code. \\

\noindent\textit{Answer: }~\url{https://godbolt.org/z/3GWE8bves} \\
